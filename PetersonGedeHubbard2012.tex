\documentclass{svjour3}                     % onecolumn (standard format)
\RequirePackage{fix-cm}

\usepackage{graphicx}
\usepackage{amsmath}
\usepackage{amssymb}
\usepackage{bm}

%% Big-O notation.
\providecommand{\OO}[1]{\ensuremath{\operatorname{O}\bigl(#1\bigr)}}

\begin{document}

\title{A linearization procedure for constrained multibody systems}

\titlerunning{A linearization procedure for constrained multibody systems}        % if too long for running head

\author{Dale L. Peterson\and Gilbert Gede\and Mont Hubbard}

\institute{Dale L. Peterson, Gilbert Gede, Mont Hubbard \at
           Sports Biomechanics Laboratory\\
           Department of Mechanical and Aerospace Engineering\\
           University of California Davis\\
           Davis, CA 95616-5294\\
           Tel.: +1 530 752 2163\\
           \email{\{dlpeterson,ggede,mhubbard\}@ucdavis.edu}}

\date{Received: date / Accepted: date}

\maketitle

\begin{abstract}
  We present a systematic procedure for linearizing equations of motion
  governing constrained multibody systems which have been derived with Kane's
  method.  The procedure avoids differential algebraic equations and yields
  linear relationships between time derivatives of generalized
  coordinates/speeds (both independent and dependent) and independent
  generalized coordinates/speeds and exogenous inputs.  These linear ODEs
  effectively embed the constraints, and enable stability analysis and control
  system design without the need to explicitly address the constraints.  We
  numerically validate the results on a well studied model of a bicycle which
  includes both configuration and velocity constraints.
\keywords{symbolic dynamics, control, constrained multibody systems}
\end{abstract}

\section{Introduction}
\label{intro}

The goal of the following calculations is to concretely establish the first
order relationships (about an arbitrary equilibrium point) between an arbitrary
choice of independent coordinates and speeds, and the time-derivatives of all
coordinates and speeds (both independent and dependent).  This is necessary for
tasks which require a linear state space formulation.  Common examples include
stability analysis, controller and estimator design (including control or
measurement of a dependent quantity).  The formalism presented here is generic
enough to cover most examples of time-varying constrained multibody systems
with arbitrary external inputs and arbitrary specified quantities.  This
procedure has been created for systems whose equations of motion have been
derived with Kane's Method.  While this is not a strict requirement, systems
which can be completely described by a set of ODEs is (i.e., no DAEs).

The complete configuration of the system is assumed to be described by
generalized coordinates $\bm{q}\in\bm{R}^n$ and its velocity completely
described by generalized speeds $\bm{u}\in\bm{R}^o$, with $n \ne o$ in general.
All coordinates and speeds, not only the independent ones, are represented by
$\bm{q}$ and $\bm{u}$.  The system is assumed to be governed by the
relationships described in Table \ref{table:assumptions}.  The system is
described by $l + m + n + o$ equations which relate $2n + 2o + s + 1$
quantities, which implies that $n - l + o - m + s + 1$ of these quantities are
independent.  This paper discusses the first order relationship between these
independent quantities ($n-l$ independent coordinates, $o-m$ independent
speeds, $s$ control inputs, and time $t$) and the time derivatives of the $n$
coordinates and $o$ speeds.
\begin{table}[htbp]
  \centering
  \caption{Constrained multibody system governing definitions and equations}
  \label{table:assumptions}
  \begin{tabular}[c]{l l l}
    Quantity & Shape & Description\\
    \hline
    $\bm{q},\bm{\dot{q}}$ & $\mathbf{R}^n$ & Coordinates and their time
    derivatives\\
    $\bm{u}, \bm{\dot{u}}$ & $\mathbf{R}^o$ & Speeds and their time derivatives\\
    $\bm{r}$ & $\mathbf{R}^s$ & Exogenous inputs \\
  \end{tabular}
  \begin{tabular}[c]{r @{ $=$ } l l}
    \multicolumn{3}{c}{ } \\
    \multicolumn{2}{c}{Equation} & Description \\
    \hline
    $\bm{f}_{c}(\bm{q}, t)$ & $\bm{0}_{l \times 1}$ & Configuration constraints \\
    $\bm{f}_{v}(\bm{q}, \bm{u}, t)$ & $\bm{0}_{m \times 1}$ & Velocity constraints \\
    $\bm{f}_{a}(\bm{q}, \bm{\dot{q}}, \bm{u}, \bm{\dot{u}}, t)$ & $\bm{0}_{m
    \times 1}$ & Acceleration constraints \\
    $\bm{f}_{0}(\bm{q}, \bm{\dot{q}}, t) + \bm{f}_{1}(\bm{q}, \bm{u}, t)$ &
    $\bm{0}_{n \times 1}$ & Kinematic differential equations \\
    $\bm{f}_{2}(\bm{q}, \bm{\dot{u}}, t) + \bm{f}_{3}(\bm{q}, \bm{\dot{q}},
    \bm{u}, \bm{r}, t)$ & $\bm{0}_{(o - m) \times 1}$ & Dynamic differential equations
  \end{tabular}
\end{table}

The first three equations in Table \ref{table:assumptions} represent
constraints derived purely from kinematic considerations at the configuration,
velocity, and acceleration levels, respectively.  The velocity constraints may
be nonholonomic or time differentiated holonomic.  The acceleration constraints
can be time differentiated velocity constraints or kinematic acceleration
constraints.  The fourth equation represents kinematic differential equations
which relate $\bm{\dot{q}}$ to $\bm{u}$ and are linear in both of these
quantities.  The fifth equation represents the constraint-free nonholonomic
dynamic differential equations~\cite{Kane1985}, and will necessarily be linear
in $\bm{\dot{u}}$ and the inputs $\bm{r}$, but generally nonlinear in all other
quantities.  This formalism is exceedingly generic; in practice, the structure
of the functions in Table \ref{table:assumptions} can be leveraged to simplify
the following calculations.  This will be discussed subsequently.

Trajectories of the system exist on a $p \triangleq n - l + o - m$ dimensional
manifold embedded in a $n + o$ dimensional space, though conserved quantities
(e.g., total mechanical energy) may restrict trajectories to a lower
dimensional manifold.  In general this cannot be assumed, especially
in the presence of external control inputs $\bm{r}$ which may change the total
mechanical energy of the system.  While more than $p$ of the independent state
variables may be of interest (i.e., dependent coordinates or speeds may be of
interest in themselves), only $p$ quantities are truly independent.  Which of
these $p$ coordinates and speeds should be chosen as independent out of the
possible $n + o$ may be obvious for small systems where there is intuition
about the system behavior, but in general the optimal choice depends on how the
configuration $\bm{q}$ affects the gradients of $f_c$ and $f_v$.  On the other
hand, in systems with cyclic/ignorable coordinates (i.e., coordinates which do
not appear in the dynamic equations of motion), fewer than $p$ state
variables may be of interest; one example is the Whipple bicycle model where
$p=10$ but 5 of these state variables are cyclic coordinates (coordinates of
rear wheel contact point, heading, and both wheel angles) and may only be of
secondary interest.

\section{Derivation of linearization algorithm}
\label{sec:derivations}

We see in Table \ref{table:assumptions} the differential equations which
describe our system. The goal of this section is to create linear equations
of motion with the following structure:
\begin{align}
    \label{eq:structure}
    \tilde{M} \dot{\bm{x}} = \tilde{A} \bm{x}_i + \tilde{B} \bm{r}
\end{align}
where $\bm{x}$ is the state vector of all the coordinates and speeds and
$\bm{x}_i$ is the state vector of only the independent coordinates and speeds.
This equation can be used for simulation, stability analysis,
or it can be reformulated into the traditional state space equations for
a variety of tasks; these topics are discussed in \ref{sec:discussion}

We begin with a first order Taylor series expansion of the equations in Table
\ref{table:assumptions} about $\bm{q}=\bm{q}^*$, $\bm{\dot{q}}=\bm{\dot{q}}^*$,
$\bm{u}=\bm{u}^*$, $\bm{\dot{u}}=\bm{\dot{u}}^*$, $\bm{r}=\bm{r}^*$; it is
assumed that all of the equations in Table \ref{table:assumptions} are
satisfied by these quantities.  In the interest of brevity, we omit writing
this equilibrium after each gradient in the calculations below; all are
evaluated at these equilibrium conditions.  Expansion of the three constraint
equations yields
\begin{align}
  \label{eq:configuration_expansion}
  \bm{f}_{c}(\bm{q}, t) &\approx \underbrace{\bm{f}_{c}(\bm{q}^*, t)}_{\bm{0}} +
  \nabla_{\bm{q}}\bm{f}_{c} \delta \bm{q}\\
  \label{eq:velocity_expansion}
  \bm{f}_{v}(\bm{q}, \bm{u}, t) &\approx \underbrace{\bm{f}_{v}(\bm{q}^*,
  \bm{u}^*, t)}_{\bm{0}} +  \nabla_{\bm{q}}\bm{f}_{v} \delta \bm{q} +
  \nabla_{\bm{u}}\bm{f}_{v} \delta \bm{u} \\
  \label{eq:acceleration_expansion}
  \bm{f}_{a}(\bm{q}, \bm{\dot{q}}, \bm{u}, \bm{\dot{u}}, t) &\approx
  \underbrace{\bm{f}_{a}(\bm{q}^*, \bm{\dot{q}}^*, \bm{u}^*, \bm{\dot{u}}^*,
t)}_{\bm{0}} +  \nabla_{\bm{q}}\bm{f}_{a} \delta \bm{q} +
\nabla_{\bm{\dot{q}}}\bm{f}_{a}
 \delta \bm{\dot{q}} \notag\\
&+ \nabla_{\bm{u}}\bm{f}_{a} \delta \bm{u} + \nabla_{\bm{\dot{u}}}\bm{f}_{a}
\delta \bm{\dot{u}}
\end{align}
The first terms are identically zero because of the assumption that the
equilibrium point satisfies the constraints.  The Taylor series expansion of
the kinematic differential equations is
\begin{align}
  \label{eq:f0_expansion}
  \bm{f}_{0}(\bm{q}, \bm{\dot{q}}, t) &\approx \bm{f}_{0}(\bm{q}^*,
  \bm{\dot{q}}^*, t) + \nabla_{\bm{q}}\bm{f}_{0} \delta\bm{q} +
  \nabla_{\bm{\dot{q}}}\bm{f}_{0} \delta\bm{\dot{q}}\\
  \label{eq:f1_expansion}
  \bm{f}_{1}(\bm{q}, \bm{u}, t) &\approx \bm{f}_{1}(\bm{q}^*,
  \bm{u}^*, t) + \nabla_{\bm{q}}\bm{f}_{1} \delta\bm{q} +
  \nabla_{\bm{u}}\bm{f}_{1} \delta\bm{u}
\end{align}
Summing (\ref{eq:f0_expansion}) and (\ref{eq:f1_expansion}) and recognizing
that the sum of the first term on the right hand side of each equation must
equal zero, we obtain
\begin{align}
  \label{eq:f0_plus_f1_expansion}
  \bm{f}_{0}(\bm{q}, \bm{\dot{q}}, t) + \bm{f}_{1}(\bm{q}, \bm{u}, t) &\approx
  \nabla_{\bm{q}}(\bm{f}_{0} + \bm{f}_{1}) \delta\bm{q} +
  \nabla_{\bm{\dot{q}}}\bm{f}_{0} \delta\bm{\dot{q}} +
  \nabla_{\bm{u}}\bm{f}_{1} \delta\bm{u}
\end{align}
Similarly, a Taylor series expansion of the dynamic differential equations, we
obtain
\begin{align}
  \label{eq:f2_expansion}
  \bm{f}_{2}(\bm{q}, \bm{\dot{u}}, t) &\approx
      \bm{f}_{2}(\bm{q}^*, \bm{\dot{u}}^*, t) +
      \nabla_{\bm{q}}\bm{f}_{2} \delta\bm{q}
      + \nabla_{\bm{\dot{u}}}\bm{f}_{2} \delta\bm{\dot{u}}\\
  \bm{f}_{3}(\bm{q}, \bm{\dot{q}}, \bm{u}, \bm{r}, t) &\approx
  \bm{f}_{3}(\bm{q}^*, \bm{\dot{q}}^*, \bm{u}^*, \bm{r}^*, t) +
  \nabla_{\bm{q}}\bm{f}_{3} \delta\bm{q}\notag\\
  \label{eq:f3_expansion}
  &+ \nabla_{\bm{\dot{q}}}\bm{f}_{3} \delta\bm{\dot{q}}
  + \nabla_{\bm{u}}\bm{f}_{3} \delta \bm{u}
  + \nabla_{\bm{r}}\bm{f}_{3} \delta\bm{r}
\end{align}
Summing (\ref{eq:f2_expansion}) and (\ref{eq:f3_expansion}) and recognizing
that the sum of the first term on the right hand sides of these equations must
equal zero, we obtain
\begin{align}
  \bm{f}_{2}(\bm{q}, \bm{\dot{u}}, t) + \bm{f}_{3}(\bm{q}, \bm{\dot{q}},
  \bm{u}, \bm{r}, t) &\approx \nabla_{\bm{q}}(\bm{f}_2 + \bm{f}_3)
  \delta\bm{q} + \nabla_{\bm{\dot{q}}}\bm{f}_{3} \delta\bm{\dot{q}}\notag\\
  \label{eq:f2_plus_f3_expansion}
  &+ \nabla_{\bm{u}}\bm{f}_{3} \delta\bm{u} +
  \nabla_{\bm{\dot{u}}}\bm{f}_{2} \delta\bm{\dot{u}} + \nabla_{\bm{r}}\bm{f}_{3} \delta\bm{r}
\end{align}

Equating the right hand sides of equations (\ref{eq:f0_plus_f1_expansion}),
(\ref{eq:acceleration_expansion}),
and (\ref{eq:f2_plus_f3_expansion}) to zero (as per Table
\ref{table:assumptions}), and introducing the following definitions
\begin{align}
    \label{eq:quant_to_compute}
    \begin{array}{llcll}
\tilde{M}_{qq}  &\triangleq \nabla_{\bm{\dot{q}}}\bm{f}_0 & \quad &
\tilde{M}_{uqc} &\triangleq \nabla_{\bm{\dot{q}}}\bm{f}_a \\
\tilde{M}_{uuc} &\triangleq \nabla_{\bm{\dot{u}}}\bm{f}_a & \quad &
\tilde{M}_{uqd} &\triangleq \nabla_{\bm{\dot{q}}}\bm{f}_2 \\
\tilde{M}_{uud} &\triangleq \nabla_{\bm{\dot{u}}}\bm{f}_2 & \quad &
\tilde{A}_{qq}  &\triangleq -\nabla_{\bm{q}}(\bm{f}_0 + \bm{f}_1) \\
\tilde{A}_{qu}  &\triangleq -\nabla_{\bm{u}}\bm{f}_1 & \quad &
\tilde{A}_{uqc} &\triangleq - \nabla_{\bm{q}} \bm{f}_a \\
\tilde{A}_{uuc} &\triangleq - \nabla_{\bm{u}} \bm{f}_a & \quad &
\tilde{A}_{uqd} &\triangleq - \nabla_{\bm{q}} (\bm{f}_2 + \bm{f}_3) \\
\tilde{A}_{uud} &\triangleq - \nabla_{\bm{u}} \bm{f}_3 & \quad &
\tilde{B}_{u}   &\triangleq -\nabla_{\bm{r}}\bm{f}_{3}
\end{array}
\end{align}
enables the unconstrained linear state space equations to be written as
\begin{align}
  \label{eq:state_space_unconstrained}
  \left[
    \begin{array}{cc}
      \tilde{M}_{qq} & \bm{0}_{n \times o} \\
      \tilde{M}_{uqc} & \tilde{M}_{uuc} \\
      \tilde{M}_{uqd} & \tilde{M}_{uud}
    \end{array}
    \right]
    \left[
      \begin{array}{c}
        \delta \bm{\dot{q}} \\
        \delta \bm{\dot{u}}
      \end{array}
    \right]
   &=
   \left[
     \begin{array}{cc}
       \tilde{A}_{qq} & \tilde{A}_{qu} \\
       \tilde{A}_{uqc} & \tilde{A}_{uuc} \\
       \tilde{A}_{uqd} & \tilde{A}_{uud}
     \end{array}
   \right]
    \left[
      \begin{array}{c}
        \delta \bm{q} \\
        \delta \bm{u}
      \end{array}
    \right]
    +
    \left[
      \begin{array}{c}
        \bm{0}_{(n + m) \times s} \\
        \tilde{B}_{u}
      \end{array}
    \right]
    \delta \bm{r}
\end{align}
Equation (\ref{eq:state_space_unconstrained}) has a state space of dimension $n
+ o$, yet only $p = n - l + o - m$ of these quantities are independent.
To address this issue, a particular set of independent
coordinates and speeds must be selected. To this end, consider the following
partitioning of the generalized coordinates and generalized speeds:
\begin{equation*}
  \tilde{\bm{q}} \triangleq \left[\begin{array}{cc}\bm{q}_{i} &
      \bm{q}_{d}\end{array}\right]^{T} =  P_{q}^{-1} \bm{q}
      \qquad\qquad
  \tilde{\bm{u}} \triangleq \left[\begin{array}{cc}\bm{u}_{i} &
      \bm{u}_{d}\end{array}\right]^{T} =  P_{u}^{-1} \bm{u}
\end{equation*}
where $P_q \in \mathbf{R}^{n \times n}$ and $P_u \in \mathbf{R}^{o \times o}$
are invertible permutation matrices which map an ordering which has the
independent quantities ($\bm{q}_{i}\in\mathbf{R}^{n-l},\,
\bm{u}_{i}\in\mathbf{R}^{o-m})$ first, followed by the dependent quantities
($\bm{q}_{d}\in\mathbf{R}^{l},\, \bm{u}_{d}\in\mathbf{R}^{m}$) to the
original ordering of the coordinates and speeds.  We use the notation $P_{qi}$
and $P_{qd}$ to denote the first $n-l$ and last $l$ columns of $P_q$,
respectively; similarly, $P_{ui}$ is the first $o-m$ columns of $P_{u}$ while
$P_{ud}$ is the last $m$ columns of $P_u$.

By assumption, equation
(\ref{eq:configuration_expansion}) is zero; making use of $P_q$, we have
\begin{align}
  \bm{0} &= \nabla_{\bm{q}}\bm{f}_{c} P_{q} \delta \bm{\tilde{q}} \notag \\
   &= \nabla_{\bm{q}}\bm{f}_{c} P_{qi} \delta \bm{q_i} +
  \nabla_{\bm{q}}\bm{f}_{c} P_{qd} \delta \bm{q_d}\notag\\
  \implies \delta \bm{q}_d &= -(\nabla_{\bm{q}}\bm{f}_{c} P_{qd})^{-1}
  (\nabla_{\bm{q}}\bm{f}_{c} P_{qi}) \delta \bm{q}_i \notag\\
  \implies \delta \bm{q} &= \left[ I_{n \times n} - P_{qd}(\nabla_{\bm{q}}
    \bm{f}_{c} P_{qd})^{-1} \nabla_{\bm{q}} \bm{f}_{c} \right] P_{qi} \delta
    \bm{q}_i
\end{align}
For convenience, we define
\begin{equation}
  \label{eq:C_0}
  C_0 \triangleq \left[ I_{n \times n} - P_{qd}(\nabla_{\bm{q}}
    \bm{f}_{c} P_{qd})^{-1} \nabla_{\bm{q}} \bm{f}_{c} \right] P_{qi}
\end{equation}
which enables us to write
\begin{align}
  \label{eq:delta_q}
  \delta \bm{q} &= C_0 \delta \bm{q}_i
\end{align}

Applying the same approach to (\ref{eq:velocity_expansion}), we obtain
\begin{align}
  \bm{0} &= \nabla_{\bm{q}}\bm{f}_{v} \delta \bm{q} +
  \nabla_{\bm{u}}\bm{f}_{v} \delta \bm{u}\notag\\
  &= \nabla_{\bm{q}} \bm{f}_{v}
\delta \bm{q} + \nabla_{\bm{u}} \bm{f}_{v} P_{ui} \delta \bm{u}_i +
\nabla_{\bm{u}} \bm{f}_{v} P_{ud} \delta \bm{u}_d \notag\\
\implies \delta \bm{u}_d &= -\left(\nabla_{\bm{u}} \bm{f}_{v}
P_{ud}\right)^{-1}\left[\nabla_{\bm{q}}\bm{f}_{v} \delta\bm{q} +
  \nabla_{\bm{u}} \bm{f}_{v} P_{ui} \delta \bm{u}_i \right]\notag\\
  \implies \delta \bm{u} &= -P_{ud}(\nabla_{\bm{u}} \bm{f}_{v} P_{ud})^{-1}
  \nabla_{\bm{q}} \bm{f}_{v} \delta \bm{q}\notag\\
  &+ \left[I - P_{ud} (\nabla_{\bm{u}}\bm{f}_{v} P_{ud})^{-1} \nabla_{\bm{u}}
    \bm{f}_{v} \right] P_{ui} \delta \bm{u}_i
\end{align}
We define
\begin{align}
  \label{eq:C_1}
  C_1 &\triangleq -P_{ud}(\nabla_{\bm{u}} \bm{f}_{v} P_{ud})^{-1}
  \nabla_{\bm{q}} \bm{f}_{v} \\
  \label{eq:C_2}
  C_2 &\triangleq \left[I - P_{ud} (\nabla_{\bm{u}}\bm{f}_{v} P_{ud})^{-1} \nabla_{\bm{u}}
    \bm{f}_{v} \right] P_{ui}
\end{align}
which enables us to write
\begin{align}
  \label{eq:delta_u}
  \delta \bm{u} &= C_1 \delta \bm{q} + C_2 \delta \bm{u}_i\notag\\
  &= C_1 C_0 \delta \bm{q}_i + C_2 \delta \bm{u}_i
\end{align}

By making use of equations (\ref{eq:delta_q}) and (\ref{eq:delta_u}), we can
rewrite equation (\ref{eq:state_space_unconstrained}) as
\begin{align}
  \label{eq:state_space_constrained}
  \left[
    \begin{array}{cc}
      \tilde{M}_{qq} & \bm{0}_{n \times o} \\
      \tilde{M}_{uqc} & \tilde{M}_{uuc} \\
      \tilde{M}_{uqd} & \tilde{M}_{uud}
    \end{array}
    \right]
    \left[
      \begin{array}{c}
        \delta \bm{\dot{q}} \\
        \delta \bm{\dot{u}}
      \end{array}
    \right]
   &=
   \left[
     \begin{array}{cc}
       (\tilde{A}_{qq} + \tilde{A}_{qu} C_1 ) C_0 & \tilde{A}_{qu} C_2 \\
       (\tilde{A}_{uqc} + \tilde{A}_{uuc} C_1 ) C_0 & \tilde{A}_{uuc} C_2\\
       (\tilde{A}_{uqd} + \tilde{A}_{uud} C_1 ) C_0 & \tilde{A}_{uud} C_2
     \end{array}
   \right]
    \left[
      \begin{array}{c}
        \delta \bm{q}_i \\
        \delta \bm{u}_i
      \end{array}
    \right]
    +
    \left[
      \begin{array}{c}
        \bm{0}_{(n+m) \times s} \\
        \tilde{B}_{u}
      \end{array}
    \right]
    \delta \bm{r}
\end{align}
We now have a linear system of equations which relates the time derivatives of all
states (dependent and independent) to the independent states and exogenous inputs.

\section{Discussion}
\label{sec:discussion}
Equation \ref{eq:state_space_constrained} doesn't fit into the standard linear
system framework because it is non-square.  To obtain a square system (useful
in stability analysis and control system design) we define
\begin{align}
  \label{eq:A_prime}
    A^\prime &\triangleq \tilde{M}^{-1} \tilde{A} \\
  \label{eq:B_prime}
    B^\prime &\triangleq \tilde{M}^{-1} \tilde{B}
\end{align}
where  $A^\prime \in \bm{R}^{(o + n) \times (o - m + n -l)}$, $B^\prime \in
\bm{R}^{(o + n) \times s}$.  We can extract the rows corresponding to the
independent states by defining
\begin{align}
  \label{eq:P_prime}
    P^\prime &\triangleq \begin{bmatrix}
        P_{qi} & \bm{O}_{n \times (o - m)} \\
        \bm{O}_{o \times (n - l)} & P_{ui}
    \end{bmatrix} \\
  \label{eq:A}
    A &\triangleq P^{\prime T} A^\prime \\
  \label{eq:B}
    B &\triangleq P^{\prime T} B^\prime
\end{align}
where $P^\prime \in \bm{R}^{(o - m + n - l) \times (o - m + n - l)}$.  Defining
$\bm{x}_i = \left[\delta\bm{q}_i,\,\delta\bm{u}_i\right]^{T}$ yields the square
state space system $\dot{\bm{x}}_i = A \bm{x}_i + B \bm{r}$.  The standard
linear systems analyses may now be applied.  It is worth noting that the rows
of $A^\prime$ and $B^\prime$ which correspond to dependent states can be used
in the output or measurement equations of a linear state space model, as would
be the case when the control task is to control a dependent state or when
measurements of dependent state(s) are made.

The choice on which state variables are taken to be independent may determine
whether $\nabla_{\bm{q}}\bm{f}_{c} P_{qd}$ and $\nabla_{\bm{u}} \bm{f}_{v}
P_{ud}$ are nonsingular.  These matrices depend upon the configuration $\bm{q}$
and constant parameters; for certain configurations or parameters, it may be
that the choice of independent state variables cannot be arbitrary.  While some
systems may permit a choice of independent state variables which are valid for
all configurations of interest, others may not.  Methods for automatically
selecting the ``best'' choice of independent state variables are discussed in
\cite{Reckdahl1996}; they involve computing the SVD of the Jacobian of the
constraint matrices to determine a set of independent speeds which will ensure
the non-singularity of the aforementioned matrices.

This algorithm as been applied to two well studied systems: the rolling disc
and an extended version of the Whipple bicycle model\cite{Meijaard2007}.
Detailed model description, derivation, and benchmarking results are presented
in the electronic supplementary material.

\section{Conclusion}
We have presented a procedure for creating linear equations of motion for
systems with holonomic and nonholonomic constraints. We have also described
some of the considerations that must be made when following this procedure as
well validating it with examples.

In summary, the procedure is:
\begin{enumerate}
    \item Describe the system in the form of the equations shown in Table
        \ref{table:assumptions}.
    \item Determine a point of linearization ($\bm{q}^*$, $\dot{\bm{q}}^*$,
      $\bm{u}^*$, $\dot{\bm{u}}^*$, and $\bm{r}^*$) which satisfies the
      equations in Table \ref{table:assumptions}.
    \item Compute the quantities in equation (\ref{eq:quant_to_compute}).
    \item Identify independent coordinates and speeds; form equations (\ref{eq:C_0}), (\ref{eq:C_1}), and
        (\ref{eq:C_2}).
    \item Form equation (\ref{eq:state_space_constrained}), and if needed,
      equations (\ref{eq:A_prime})-(\ref{eq:B}).
\end{enumerate}

\begin{acknowledgements}
  This material is based upon work partially supported by the National Science
  Foundation under award 0928339 and two Google Summer of Code projects (2009,
  2011).  Jason Moore, Thomas Johnston, Evan Sperber, and Andrew Kickertz
  provided valuable feedback during discussions of multibody dynamics and
  control.
\end{acknowledgements}

% Not sure which bibliography style we are supposed to use
%\bibliographystyle{spphys}       % APS-like style for physics
%\bibliographystyle{spbasic}      % basic style, author-year citations
\bibliography{references}   % name your BibTeX data base
\bibliographystyle{spmpsci}      % mathematics and physical sciences
\end{document}
