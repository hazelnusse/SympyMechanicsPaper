\documentclass{article}

\begin{document}
\section{Rolling disc with skate attached to arm}
Consider a thin disc of mass $m$, radius $r$, radial inertia $mr^2/2$, and,
transverse inertia $mr^2/4$.  At the disc center is an axle attached to one end
of a massless rod of length $l$; to the other end of the rod a skate is
affixed.  The skate permits frictionless sliding along the line of intersection
between the disc plane and the ground plane upon which the disc rolls without
slip.  The skate permits no sliding direction perpendicular to this line.

The inertial frame $N$ has $n_z$ aligned with the local gravitational field,
while $n_x$ and $n_y$ span the ground plane with $n_z = n_x \times n_y$.  Four
generalized coordinates $q_0, q_1, q_2, q_3$ sequentially orient four frames
$A, B, C, D$.  $A$ is oriented relative to $N$ by a $q_0$ rotation about $n_z$;
this is the yaw frame.  $B$ is oriented relative to $A$ by a $q_1$ rotation
about $a_x$; this is the lean frame. $C$ and is oriented relative to $B$ by a
$q_2$ rotations about $b_y$; this is the disc frame ($c_y$ is perpendicular to
the disc plane).  $D$ is oriented relative to $B$ by a $q_3$ rotation about
$b_y$; this is the arm frame ($d_z$ is parallel to the line between the disc
center and the arm-ground contact point).

A torque $\tau$ is applied between the disc and the arm by means of a motor.


\end{document}

