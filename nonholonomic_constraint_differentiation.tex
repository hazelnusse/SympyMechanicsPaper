\documentclass[letterpaper,11pt]{article}
\usepackage[round]{natbib}
\usepackage[margin=1in,centering]{geometry}
\usepackage{fancyhdr}
\usepackage{amsmath}
\usepackage{amssymb}
\usepackage{graphicx}
\usepackage[pdftex]{hyperref}
\hypersetup{
    pdftitle={Nonholonomic constraint time differentation},
    pdfauthor={Dale Lukas Peterson},
    pdfsubject={Subject},
    pdfkeywords={Keywords}}

\pagestyle{fancy}
\fancyhead[L]{}
\fancyhead[R]{}  % page number on the right
\fancyfoot[L,C,R]{}  %  No footer on left, center or right, on even or odd pages

\begin{document}
\begin{abstract}
  When linear velocity constraints are encountered in the modelling of a
  mechanical system, it is generally possible to find a set of independent
  speeds which allow for these constraints to be solved for the remaining
  dependent speeds in terms of the independent speeds.  Under these conditions
  we investigate whether time differentation of the motion constraints, and
  subsequently solving these time differentiated equations for the time
  derivatives of the dependent speeds yields the same results as if the
  dependent speeds are differentiated directly.  We find that they are equivalent, a finding that has practical implications for symbolic multibody
  dynamic codes.
\end{abstract}

\section{Problem statement}
Assume the constraints are of the form
\begin{equation}
  \label{eq:velocity_constraints}
  B u = 0
\end{equation}
where $u\in\mathbf{R}^o$, and $B \in \mathbf{R}^{m \times o}$.  Assume that $u
= \left[\begin{array}{cc}u_i & u_d\end{array}\right]^{T}$, i.e., that $o-m$
  independent speeds come first, followed by the $m$ dependent speeds.  We can
  then rewrite equation \ref{eq:velocity_constraints} as
\begin{align}
  \label{eq:BiuiBdud}
  B_i u_i + B_d u_d &= 0 \\
  \label{eq:ud}
  \implies u_d &= -B_d^{-1} B_i u_i
\end{align}

Direct differentiation of equation \ref{eq:BiuiBdud} yields
\begin{align}
  \frac{d}{dt} \left(B_i u_i + B_d u_d \right) &= \frac{d}{dt} 0 \\
  \dot{B}_i u_i + B_i \dot{u}_i + \dot{B}_d u_d + B_d \dot{u}_d &= 0 \\
  \implies \dot{u}_d &= -B_d^{-1} \left(\dot{B}_i u_i + B_i \dot{u}_i + \dot{B}_d u_d \right)\\
  &= -B_d^{-1} \left(\dot{B}_i u_i + B_i \dot{u}_i+ \dot{B}_d \left(-B_d^{-1} B_i u_i \right) 
  \right) \\
  &= -B_d^{-1} \left( \dot{B}_i - \dot{B}_d B_d^{-1} B_i \right) u_i - B_d^{-1}
  B_i \dot{u}_i
\end{align}
In contrast, direct differentiation of equation \ref{eq:ud} yields
\begin{align}
  \frac{d}{dt} u_d &= -\frac{d}{dt}\left(B_d^{-1} B_i u_i\right)\\
\implies  \dot{u}_d &= -\frac{d}{dt}\left(B_d^{-1} B_i \right) u_i -
  \left(B_d^{-1} B_i \right) \frac{d}{dt}\left(u_i\right)\\
  &= -\dot{B}_d^{-1} B_i u_i - B_d^{-1} \dot{B}_i u_i
                    - B_d^{-1} B_i \dot{u}_i \\
  &= -\left(\dot{B}_d^{-1} B_i + B_d^{-1} \dot{B}_i\right) u_i - B_d^{-1} B_i \dot{u}_i
\end{align}

The latter terms of both versions of $\dot{u}_d$ are clearly equal.  For the
first term to be equal, we must then have

\begin{align}
-B_d^{-1} \left( \dot{B}_i - \dot{B}_d B_d^{-1} B_i \right) &=
-\left(\dot{B}_d^{-1} B_i + B_d^{-1} \dot{B}_i\right) \\
\implies B_d^{-1} \dot{B}_d B_d^{-1} B_i &= -\dot{B}_d^{-1} B_i\\
\implies B_d^{-1} \dot{B}_d B_d^{-1} &= -\dot{B}_d^{-1} \\
\implies \dot{B}_d B_d^{-1} + B_d \dot{B}_d^{-1} &= 0 \\
\implies \frac{d}{dt} \left(B_d B_d^{-1} \right) &= 0 \\
\implies \frac{d}{dt} I &= 0
\end{align}

\end{document}
